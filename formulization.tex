\documentclass[11pt]{article}
\usepackage{amsmath, amssymb}
\usepackage[margin=1in]{geometry}
\usepackage{booktabs}

\title{Mathematical Formulation: Milk Bank Collection Optimization}
\author{Oklahoma Mothers' Milk Bank}
\date{}

\begin{document}
\maketitle

\section{Problem Description}

The Oklahoma Mothers' Milk Bank operates a network of 36 depots where donors drop off frozen breast milk. A single driver must collect milk from a subset of these depots before their freezers reach capacity. The objective is to minimize total travel time while ensuring each depot is visited exactly once per route.

This problem is formulated as a Traveling Salesman Problem (TSP) for each pickup route. We solve two separate TSP instances: one for the Local (OKC Metro) depots and one for the Tulsa depots. The remaining 22 depots are serviced via courier shipping and are not included in the routing model.

\section{Sets}

\begin{align*}
N &= \text{Set of all nodes (depots) in a route, indexed } i, j \in \{0, 1, 2, ..., n-1\} \\
&\quad \text{where node } 0 \text{ represents the Milk Bank headquarters (HQ)}
\end{align*}

\textbf{Route sizes:}
\begin{itemize}
    \item Local route: $|N| = 9$ (HQ + 8 depots)
    \item Tulsa route: $|N| = 7$ (HQ + 6 depots)
\end{itemize}

\section{Parameters}

\begin{align*}
t_{ij} &= \text{Travel time (in minutes) from depot } i \text{ to depot } j, \quad \forall i, j \in N \\
n &= |N| = \text{Total number of nodes in the route}
\end{align*}

\section{Decision Variables}

\begin{align*}
x_{ij} &= 
\begin{cases} 
1 & \text{if the driver travels directly from depot } i \text{ to depot } j \\
0 & \text{otherwise}
\end{cases}
\quad \forall i, j \in N \\[10pt]
u_i &= \text{Position of depot } i \text{ in the route (for subtour elimination)}, \quad \forall i \in N \setminus \{0\}
\end{align*}

\section{Objective Function}

Minimize total travel time:

\begin{equation}
\min \sum_{i \in N} \sum_{j \in N} t_{ij} \cdot x_{ij}
\end{equation}

\section{Constraints}

\textbf{1. Leave each depot exactly once:}
\begin{equation}
\sum_{j \in N, j \neq i} x_{ij} = 1, \quad \forall i \in N
\end{equation}

\textbf{2. Enter each depot exactly once:}
\begin{equation}
\sum_{i \in N, i \neq j} x_{ij} = 1, \quad \forall j \in N
\end{equation}

\textbf{3. No self-loops:}
\begin{equation}
x_{ii} = 0, \quad \forall i \in N
\end{equation}

\textbf{4. Subtour elimination (Miller-Tucker-Zemlin formulation):}
\begin{equation}
u_i - u_j + n \cdot x_{ij} \leq n - 1, \quad \forall i, j \in N \setminus \{0\}, \quad i \neq j
\end{equation}

\textbf{5. Variable bounds:}
\begin{align}
x_{ij} &\in \{0, 1\}, \quad \forall i, j \in N \\
u_i &\geq 0, \quad \forall i \in N \setminus \{0\}
\end{align}

\section{Constraint Explanation}

\begin{itemize}
    \item \textbf{Constraints (2) and (3)} ensure that the driver leaves and enters each depot exactly once, forming a complete tour.
    
    \item \textbf{Constraint (4)} prevents the driver from visiting the same depot twice (no self-loops).
    
    \item \textbf{Constraint (5)} is the Miller-Tucker-Zemlin (MTZ) subtour elimination constraint. Without this, the model might produce disconnected loops instead of a single tour. The variable $u_i$ tracks the order in which depots are visited. If $x_{ij} = 1$ (we travel from $i$ to $j$), then $u_j$ must be at least $u_i + 1$, ensuring a connected sequence.
\end{itemize}

\section{Model Application}

This formulation is solved twice:

\begin{table}[h]
\centering
\begin{tabular}{llcc}
\toprule
\textbf{Route} & \textbf{Depots} & \textbf{Optimal Driving Time} & \textbf{Frequency} \\
\midrule
Local (OKC Metro) & 8 + HQ & 167.8 min (2.8 hrs) & Twice weekly \\
Tulsa & 6 + HQ & 300.2 min (5.0 hrs) & Weekly \\
\bottomrule
\end{tabular}
\end{table}

\section{Total Route Time Calculation}

The total time for each route includes driving time plus service time at each stop:

\begin{equation}
T_{total} = \sum_{i \in N} \sum_{j \in N} t_{ij} \cdot x_{ij} + |N \setminus \{0\}| \cdot s
\end{equation}

where $s = 20$ minutes is the service time per depot.

\begin{table}[h]
\centering
\begin{tabular}{lccc}
\toprule
\textbf{Route} & \textbf{Driving} & \textbf{Service} & \textbf{Total} \\
\midrule
Local & 2.8 hrs & 2.7 hrs & 5.5 hrs \\
Tulsa & 5.0 hrs & 2.0 hrs & 7.0 hrs \\
\bottomrule
\end{tabular}
\end{table}

\section{Weekly Schedule Constraint}

The driver is limited to 12 hours per day. With our solution:

\begin{align*}
\text{Monday (Local):} \quad & 5.5 \text{ hrs} \leq 12 \text{ hrs} \quad \checkmark \\
\text{Wednesday (Tulsa):} \quad & 7.0 \text{ hrs} \leq 12 \text{ hrs} \quad \checkmark \\
\text{Thursday (Local):} \quad & 5.5 \text{ hrs} \leq 12 \text{ hrs} \quad \checkmark
\end{align*}

Total weekly driver time: $5.5 + 7.0 + 5.5 = 17.9$ hours.

\end{document}