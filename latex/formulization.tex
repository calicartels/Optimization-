\documentclass[9pt]{article}
\textheight  = 21cm
\textwidth   = 17cm
\topmargin = -1 cm
\oddsidemargin = -0.5cm % Adjust to change left margin
\evensidemargin = -0.5cm % For double-sided documents



\makeatletter
\setlength{\@fptop}{0pt}
\makeatother

\usepackage{booktabs}

\usepackage[table,xcdraw]{xcolor}
\usepackage{gensymb}
\usepackage{amsmath}
\usepackage[utf8]{inputenx}
\usepackage[english]{babel}
\usepackage{newunicodechar}
\newunicodechar{≈}{\approx}
\usepackage[pdftex]{graphicx} 	%  uso de graficos en general
\usepackage{subfigure} 		%  poder poner dos graficos como parte de uno
\usepackage{fancyhdr} 		%  encabezado diferente para pag pares e impares
\usepackage{epstopdf} 		%  Convertir .eps a .pdf (si fuera necesario)
\usepackage{titling}
\usepackage{fancyvrb,fancybox,calc} 
\usepackage[svgnames]{xcolor} 
\usepackage{enumerate}
\usepackage{adjustbox}
\usepackage{tabu}
\usepackage{multicol}
\usepackage{multirow}
\usepackage{float}


\usepackage{caption}
\usepackage{subcaption}

\usepackage{subfigure}
\usepackage{graphicx}
\usepackage{comment}

\captionsetup{justification=centerlast,labelfont=bf}

\newcommand{\HRule}{\rule{\linewidth}{0.5mm}}

\DeclareGraphicsExtensions{.pdf,.png,.jpg} % busca en este orden! 

\parindent=0mm

\definecolor{light-gray}{gray}{0.94}
\definecolor{textgray}{gray}{0.5}
\newenvironment{verbcode}{\VerbatimEnvironment
  \begin{Sbox} \color{textgray}\scriptsize
  \begin{minipage}{\linewidth+2\fboxsep-2\fboxrule-12pt}    
  \begin{Verbatim}
}{\end{Verbatim}  
  \end{minipage}   
  \end{Sbox} \color{black}\normalsize
  \fcolorbox{black}{light-gray}{\TheSbox} 
} 


\setlength{\parindent}{2.5em}
\setlength{\parskip}{1em}

\usepackage[colorinlistoftodos]{todonotes}

\usepackage[utf8]{inputenc}
\usepackage{amsmath}
\usepackage{amssymb}
\usepackage{xcolor}
\usepackage{listings}
\usepackage{xstring}

\usepackage{hyperref}
\hypersetup{
    colorlinks=true,
    linkcolor=black,
    filecolor=magenta,      
    urlcolor=blue,
    citecolor=black,
    pdftitle={Overleaf Example},
    pdfpagemode=FullScreen,
    }

\urlstyle{same}

\definecolor{dkgreen}{rgb}{0,0.6,0}
\definecolor{ltgray}{rgb}{0.5,0.5,0.5}

\makeatletter
\newif\ifcolname
\colnamefalse

\def\keywordcheck{%
\IfStrEq*{\the\lst@token}{select}{\global\colnametrue}{}%
\IfStrEq*{\the\lst@token}{where}{\global\colnametrue}{}%
\IfStrEq*{\the\lst@token}{from}{\global\colnamefalse}{}%
\color{blue}%
}
\def\setidcolor{%
\ifcolname\color{purple}\else\color{black}\fi%
}
\makeatother

\lstset{%
    backgroundcolor=\color{white},
    basicstyle=\footnotesize,
    breakatwhitespace=false,
    breaklines=true,
    captionpos=b,
    commentstyle=\color{dkgreen},
    deletekeywords={...},
    escapeinside={\%*}{*)},
    extendedchars=true,
    frame=single,
    keepspaces=true,
    language=SQL,
    otherkeywords={is},
    morekeywords={*,modify,MODIFY,...},
    keywordstyle=\keywordcheck,
    identifierstyle=\setidcolor,
    numbers=left,
    numbersep=15pt,
    numberstyle=\tiny,
    rulecolor=\color{ltgray},
    showspaces=false,
    showstringspaces=false, 
    showtabs=false,
    stepnumber=1,
    tabsize=4,
    title=\lstname
}

\usepackage{color}
\definecolor{dkgreen}{rgb}{0,0.6,0}
\definecolor{gray}{rgb}{0.5,0.5,0.5}
\definecolor{red}{rgb}{1,0,0}
\lstset{language=SQL,
  basicstyle={\small\ttfamily},
  belowskip=3mm,
  breakatwhitespace=true,
  breaklines=true,
  classoffset=0,
  columns=flexible,
  commentstyle=\color{dkgreen},
  framexleftmargin=0.25em,
  frameshape={}{yy}{}{}, %To remove to vertical lines on left, set `frameshape={}{}{}{}`
  keywordstyle=\color{blue},
  numbers=left, %If you want line numbers, set `numbers=left`
  numberstyle=\tiny\color{gray},
  showstringspaces=false,
  stringstyle=\color{red},
  tabsize=3,
  xleftmargin =1em
}

\usepackage{graphicx}
\usepackage{siunitx}
\usepackage{hyperref}
\usepackage{xcolor}
\usepackage{subfig}


%%%%%%%%%%%%%%%%%%%%%%%%%%%%%%%%%%%%%%%%%%%%%%%%%%%%%%%%%%%%%%%%%%
%%%%%%%%%%%%%%%%%%%%%%%%%%%%%%%%%%%%%%%%%%%%%%%%%%%%%%%%%%%%%%%%%%
%Fill in the appropriate information below
\newcommand{\norm}[1]{\left\lVert#1\right\rVert}     
\newcommand\course{16.86 - Práctica Laboral}                            % <-- course name   
\newcommand\hwnumber{AIPI 530 - Optimization in Practice}                                 % <-- homework number
\newcommand\Information{\makebox[\linewidth][c]{\parbox[t]{5cm}{\centering Vishnu Mukundan \\ Sansa Zhao }}}

                 % <-- personal information
%%%%%%%%%%%%%%%%%%%%%%%%%%%%%%%%%%%%%%%%%%%%%%%%%%%%%%%%%%%%%%%%%%
%%%%%%%%%%%%%%%%%%%%%%%%%%%%%%%%%%%%%%%%%%%%%%%%%%%%%%%%%%%%%%%%%%

\begin{document}

\thispagestyle{empty}

\begin{titlepage}
    \begin{center}
        \vspace*{3cm}
            
        \Huge
        \textbf{Final Project: Milk Bank Case Study}
            
        \vspace{1cm}
        \huge
        \hwnumber
            
        \vspace{1.5cm}
        \Large
            
        \textbf{\Information}                      % <-- author
        
            
        \vfill

        \today

        \vfill
        
        % \course
            
        \vspace{0.5cm}
            
        \includegraphics[width=0.4\textwidth]{Imagenes/logo.png}
        \\
        
        \Large
        
        
            
    \end{center}
\end{titlepage}

\pagestyle{fancy}
\headheight=60pt 	%para cambiar el tamaño del encabezado
\fancyhead[L]	{ }					
\fancyhead[R] 	{ Milk Bank Case Study }



\newpage
\section{Problem Description}

The Oklahoma Mothers' Milk Bank operates a network of 36 depots where donors drop off frozen breast milk. A single driver must collect milk from a subset of these depots before their freezers reach capacity. The objective is to minimize total travel time while ensuring that each depot is visited exactly once per route. 

\vspace{3mm} 

\noindent This problem is formulated as a Traveling Salesman Problem (TSP) for each pickup route. We solve two separate TSP instances: one for the Local (OKC Metro) depots and one for the Tulsa depots. The remaining 22 depots are serviced by courier and are not included in the routing model.

\section{Compact mathematical formulation}


\subsection{Decision Variables}

\begin{equation*}
  \text{depots of Routes}  = \text{ \{0,1,\dots,n-1\}},
\end{equation*}
where depot \(0\) denotes the Oklahoma Mothers' Milk Bank (the depot / start\&end location) and the remaining indices denote pickup depots.
\begin{align*}
x_{ij} &= 
\begin{cases} 
1 & \text{if the driver travels directly from depot } i \text{ to depot } j \\
0 & \text{otherwise}
\end{cases}
\quad \forall i, j \in N \\[10pt]
u_i &= \text{Position of depot } i \text{ in the route (for subtour elimination)}, \quad \forall i \in N \setminus \{0\}
\end{align*}
\begin{equation*}
u_i \quad \forall i \in \mathcal{N}
\end{equation*}
\begin{equation*}
u_i \in [0,\, n-1]\quad\text{(continuous)}.
\end{equation*}
\(\;u_i\) are auxiliary (order) variables used in the Miller–Tucker–Zemlin (MTZ) subtour elimination formulation: they capture the position of depot \(i\) in the tour (except that \(u_0\) is unconstrained by the MTZ inequalities below).


\subsection{Objective function:}

Minimize total travel time:

\begin{equation}
\min \sum_{i \in N} \sum_{j \in N} t_{ij} \cdot x_{ij}
\end{equation}

\newpage
\subsection{Parameters: }

% You can choose which line to use for your project, and remove the rest
\[
\begin{aligned}
    TravelTime_{ij} &\quad \forall\, i,j \in N 
\end{aligned}
\]
\[
\begin{aligned}
    n &= |N| = \text{Total number of depots in the route}
\end{aligned}
\]
\begin{itemize}
    \item Local route: $|N| = 9$ (HQ + 8 depots)
    \item Tulsa route: $|N| = 7$ (HQ + 6 depots)
\end{itemize}


\subsection{Constraints}

\paragraph{Leave each depot exactly once}
\begin{equation}
    \sum_{j\in\mathcal{N},\,j\neq i} x_{ij} = 1 \qquad \forall i\in\mathcal{N}
\end{equation}
(Each depot has exactly one outgoing arc.)

\paragraph{Enter each depot exactly once}
\begin{equation}
    \sum_{i\in\mathcal{N},\,i\neq j} x_{ij} = 1 \qquad \forall j\in\mathcal{N}
\end{equation}
(Each depot has exactly one incoming arc.)

\paragraph{No self-loops}
\begin{equation}
    x_{ii} = 0 \qquad \forall i\in\mathcal{N}
\end{equation}

\paragraph{MTZ subtour elimination (Miller--Tucker--Zemlin)}
For all \(i,j\in\mathcal{N}\) with \(i\neq 0,\;j\neq 0,\;i\neq j\):
\begin{equation}
    u_i - u_j + n\,x_{ij} \le n - 1
\end{equation}
These inequalities forbid nontrivial subtours by enforcing a consistent ordering \(u_i\) for visited depots (the factor \(n\) is the usual MTZ multiplier for a model with \(n\) depots).

\paragraph{Variable Bounds}
\begin{align}
x_{ij} &\in \{0, 1\}, \quad \forall i, j \in N \\
u_i &\geq 0, \quad \forall i \in N \setminus \{0\}
\end{align}

\newpage
\subsection{Notes on instances}

We solve two separate instances of the above model:

\begin{itemize}
    \item \textbf{Local (OKC Metro)} with \(n_{\text{local}} = 9\) nodes (node indices \(0\) to \(8\)).
    \item \textbf{Tulsa} with \(n_{\text{tulsa}} = 8\) nodes (node indices \(0\) to \(7\)).
\end{itemize}

\noindent Below we list the node index mapping used in the Local instance and the Tulsa instance (index 0 = depot).

\subsection{Local instance: node list}
\begin{align*}
 0 &: \text{Oklahoma Mothers' Milk Bank (depot)}\\
 1 &: \text{Variety Care- Lafayette}\\
 2 &: \text{Cleveland County Moore}\\
 3 &: \text{Edmond Hope Center}\\
 4 &: \text{Integris Southwest}\\
 5 &: \text{Integris Baptist}\\
 6 &: \text{St. Anthony Shawnee Hospital}\\
 7 &: \text{Canadian Valley}\\
 8 &: \text{OU Children's Hospital}
\end{align*}

\subsection{Tulsa instance: node list}
\begin{align*}
 0 &: \text{Oklahoma Mothers' Milk Bank (depot)}\\
 1 &: \text{YWCA South}\\
 2 &: \text{Tulsa Hillcrest Hospital}\\
 3 &: \text{Jane Phillips Medical Center}\\
 4 &: \text{St. John Owasso}\\
 5 &: \text{YWCA East}\\
 6 &: \text{OBI}\\
 7 &: \text{(implicit return to depot)}
\end{align*}

\newpage
\subsection{Results (from the Pyomo)}

\paragraph{Local route (optimal):}
\begin{verbatim}
 0: Oklahoma Mothers' Milk Bank (Start)
 1: Edmond Hope Center
 2: Integris Baptist
 3: Canadian Valley
 4: Integris Southwest
 5: Variety Care- Lafayette
 6: Cleveland County Moore
 7: St. Anthony Shawnee Hospital
 8: OU Children's Hospital
 9: Oklahoma Mothers' Milk Bank (Return)
\end{verbatim}
Total driving time: \(167.8\) minutes (≈ \(2.8\) hours).

\paragraph{Tulsa route (optimal):}
\begin{verbatim}
 0: Oklahoma Mothers' Milk Bank (Start)
 1: YWCA South
 2: Tulsa Hillcrest Hospital
 3: Jane Phillips Medical Center
 4: St. John Owasso
 5: YWCA East
 6: OBI
 7: Oklahoma Mothers' Milk Bank (Return)
\end{verbatim}
Total driving time: \(300.3\) minutes (≈ \(5.0\) hours).

\end{document}




